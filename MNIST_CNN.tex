\documentclass{article}

\title{Convolutional Neural Network for MNIST Dataset \\ Hyperparameter Testing and Experiments on the Trained Models}
\author{Andrei-Cristian Ilies}
\date{}

\begin{document}

\maketitle

\section*{Introduction}
\large{\hspace{1cm}MNIST is a dataset containing tiny grey-scale images, each showing a handwritten digit. The purpose of this project is to train a convolutional neural network to accurately predict the digits in the images from this dataset. Throughout this project, I tested different hyperparameters to better understand how they function in the network, and I observed the performance of all models during both training and testing on the images. In the final model, I added an additional convolutional layer to see how it would impact the network's performance.}

\section*{Workspace and Project Components}
\large{\hspace{1cm}I worked in Google Colab and trained a total of 16 models. Each model was saved to Google Drive after training and subsequently loaded and tested on randomly selected images from the test set.}

\large{\hspace{0.5cm}The project includes two .ipynb files:}

\begin{itemize}
  \item The first file (MNIST\_Hyperparameter\_Testing.ipynb) includes the training of the 16 models and saving each model in a folder on Google Drive.
  \item The second file (MNIST\_TrainedModels\_Experiments.ipynb) involves loading each model and testing it on 10 randomly selected images from the test set.
\end{itemize}

\section*{Hyperparameter Testing and Trained Models Experiments}
\large{\hspace{1cm}In this section, I will present my observations for each model individually.}

\subsection*{Model 1 - Learning Rate = 0.5}
\begin{itemize}
  \item The losses decrease very slowly, having higher values.
  \item The accuracies are very low.
  \item It has poor performance.
  \item It incorrectly predicts the digit values in the images
\end{itemize}

\subsection*{Model 2 - Learning Rate = 0.1}
\begin{itemize}
  \item The losses decrease rapidly, with values very close to 0.
  \item The accuracy increases and reaches high values.
  \item It correctly predicts the digit values in the images
\end{itemize}

\subsection*{Model 3 - Learning Rate = 0.01}
\begin{itemize}
  \item The losses decrease rapidly, with values very close to 0.
  \item The model with 0.1 learning rate has lower loss values than this model
  \item The accuracy increases and reaches high values.
  \item It correctly predicts the digit values in the images
\end{itemize}

\subsection*{Model 4 - Learning Rate = 0.001}
\begin{itemize}
  \item The losses have higher values at the start of the first epoch but decrease over time, not as rapidly as when lr = 0.01.
  \item The accuracy does not start with very high percentages, but the values increase over time.
  \item The final epoch has values similar to those of the first epoch when lr = 0.01.
  \item It correctly predicts the digit values in the images
\end{itemize}

\subsection*{Model 5 - Learning Rate = 0.0001}
\begin{itemize}
  \item The losses decrease slowly, remain at high values, and do not approach 0.
  \item The accuracy is very low.
  \item The losses have higher values compared to when lr = 0.01.
  \item It sometimes incorrectly predicts the digit values in the images.
\end{itemize}

\subsection*{Model 6 - ReLU Activation Function}
\begin{itemize}
  \item The losses decrease rapidly, with values very close to 0.
  \item The accuracy increases and reaches high values.
  \item It correctly predicts the digit values in the images.
\end{itemize}

\subsection*{Model 7 - Tanh Activation Function}
\begin{itemize}
  \item It has reasonable losses, but not as good as those from training with the ReLU activation function.
  \item The accuracy starts at a lower value compared to the model using ReLU.
  \item It correctly predicts the digit values in the images.
\end{itemize}

\subsection*{Model 8 - Sigmoid Activation Function}
\begin{itemize}
  \item The losses have values that are too high and do not approach 0 at all.
  \item The accuracy values start very low but increase significantly from epoch 4, approaching ideal values.
  \item No accuracy value surpasses those from training with the ReLU and Tanh activation functions, but the values increase substantially over time.
  \item It sometimes incorrectly predicts the digit values in the images.
\end{itemize}

\subsection*{Model 9 - PReLU Activation Function}
\begin{itemize}
  \item The loss values are slightly lower compared to ReLU.
  \item The accuracy is slightly higher compared to ReLU.
  \item It performs better than the model with the ReLU activation function.
  \item It correctly predicts the digit values in the images.
\end{itemize}

\subsection*{Model 10 - Batch Norm for each Convolutional Layer}
\begin{itemize}
  \item It has significantly better loss values from the first epoch compared to the model without batch normalization for each layer.
  \item It starts with much better accuracy.
  \item It performs better than the model without batch normalization.
  \item It correctly predicts the digit values in the images.
\end{itemize}

\subsection*{Model 11 - 50\% Dropout on the First Fully Connected Layer}
\begin{itemize}
  \item It has low loss values.
  \item It has high accuracy.
  \item It correctly predicts the digit values in the images.
\end{itemize}

\subsection*{Model 12 - 10\% Dropout on the First Fully Connected Layer}
\begin{itemize}
  \item It starts with a higher loss, but better than dropout = 50\%.
  \item The losses have better values compared to the model with dropout = 50\%.
  \item It correctly predicts the digit values in the images.
\end{itemize}

\subsection*{Model 13 - 90\% Dropout on the First Fully Connected Layer}
\begin{itemize}
  \item It starts with very high loss values.
  \item It does not perform as well as the models with lower dropout.
  \item It sometimes incorrectly predicts the digit values in the images.
\end{itemize}

\subsection*{Model 14 - RMSProp Optimizer}
\begin{itemize}
  \item Very high loss values.
  \item Low accuracy.
  \item Incorrectly predicts the digit values in the images.
\end{itemize}

\subsection*{Model 15 - Adam Optimizer}
\begin{itemize}
  \item The loss values are reasonable.
  \item The losses are higher compared to the model using SGD optimizer.
  \item The accuracy is high but not as high as the model trained with SGD.
  \item It performs better than RMSProp but not as well as SGD.
  \item It correctly predicts the digit values in the images.
\end{itemize}

\subsection*{Model 16 - Adding a Third Convolutional Layer}
\begin{itemize}
  \item Initially, it has higher loss values compared to the model with 2 layers.
  \item The loss values approach lower values over time.
  \item The accuracy is lower than in the model with 2 layers.
  \item Starting from the second epoch, the model has lower loss values.
  \item The accuracy increases and reaches high values.
  \item In the final epoch, it has better loss and accuracy values compared to the model with 2 layers.
  \item It correctly predicts the digit values in the images.
\end{itemize}

\end{document}